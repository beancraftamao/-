% -*- coding: utf-8 -*-
% !TEX program = xelatex
\documentclass[12pt, a4paper, oneside, UTF8]{ctexbook}
\usepackage{amsmath,xeCJK,hyperref,xcolor,comment,indentfirst,titlesec}
\usepackage{geometry}
\setCJKmainfont{KaiTi}
\setmainfont{Times New Roman}
\titleformat{\chapter}{\centering\bfseries\zihao{2}}{第\,\thechapter\,章}{1em}{}
\titleformat{\section}{\bfseries\zihao{4}}{\thesection}{1em}{}

\geometry{
    a4paper,
    left=3cm,
    right=3cm,
    top=3cm,
    bottom=1cm,
    headsep=1cm,
    footskip=1cm,
}
\begin{document}

\title{重生之每天坚持学不会从入学到入土\\组内自用手搓讲义}
\author{阿毛}
\date{\today}
\maketitle

\chapter{二次量子化}
二次量子化的形式可以从高量角度理解,也可以直接从场论出发,由于未学习场论,本章采用高量的语言。

\section{速通本章}
波函数语言在描述量子体系的时候是自洽的,但是在多体系统中,通常会出现
$\sum_N \int\Psi_N^{\dag}(r)\Psi_N(r) d(\vec{r})$的情况,共有$N!\times N!$项,计算量巨大。相较之下,
粒子数表示的表达非常简洁,且也能在多体系统中描述体系的变化(粒子的产生与湮灭)。
无论是费米子系统还是玻色子系统,都可以从波函数表示转化为粒子数表示
,这就是所谓的二次(历史上时间的第二次)量子化。

单体力学量算符$\hat{F}$(例如动量)在粒子数表象下可表示为:
\begin{eqnarray*}
    \hat{F}=\sum_{\alpha,\beta}f_{\alpha\beta}\hat{a}^{\dag}_\alpha\hat{a}_\beta,
\end{eqnarray*}
其中
\begin{eqnarray*}
    f_{\alpha\beta}=\langle\psi_\alpha|\hat{f}|\psi_\beta\rangle=\int dq \psi^{*}_\alpha(q)\hat{f}(q)\psi_\beta(q),
\end{eqnarray*}
而两体力学量算符
\begin{align}
    \hat{G}=&\sum_{a<b}\hat{g}(a,b)
\end{align}
例如:
\begin{align}
    V(q_1,q_2,q_N)=&\sum_{i<j}\frac{e^2}{|\mathbf{r}_i-\mathbf{r}_j|}=
    \frac{1}{2}\sum_{i,j}\frac{e^2}{|\mathbf{r}_i-\mathbf{r}_j|}
\end{align}
在粒子数表象中可以写为如下形式:
\begin{align}
    \hat{G}=\frac{1}{2}\sum_{\alpha'\beta'}\sum_{\alpha\beta}
    g_{\alpha'\beta',\alpha\beta}\hat{a}^{\dag}_{\alpha'}\hat{\alpha}^{\dag}_{\beta'}\hat{a}_\beta\hat{a}_\alpha
\end{align}
其中
\begin{eqnarray}
    g_{\alpha'\beta',\alpha\beta}=\int dq_1 dq_2 \varphi^*_{\alpha'}(q_1)\varphi^*_(q_2)\hat{g}(q_1,q_2)\varphi_\alpha(q_1)\varphi_\beta(q_2).
\end{eqnarray}
波函数$\Psi$则可以表示为粒子数的态$|n\rangle$,其中n是某个态的粒子数。
关于产生湮灭算符的作用,后面会提到,需要注意力学量作用在态上时,需要遵循火车到站的规则(老田的课里有),即算符作用后,存在一个置换的关系。


\section{数学准备}
首先明确本讲义目标:快速上手理解课题组工作中涉及的概念,如需要完整学习,详情参考B站田光善老师高等量子力学课程。
\href{https://www.bilibili.com/video/BV1qk4y1173i?p=1&vd_source=cd4587e54f9d57093a8dbe5712bb9c5e}
{\textcolor{blue}{网页链接}}

\subsection{置换群}
群论在物理学中的应用很广泛,但此处并不深入讨论,只需要把群操作视为一种算符,以方便理解。如群的定义,群的性质等,此处不说明。

置换定义如下:
\begin{align}
    \hat{P}=\left(\begin{matrix}
        1 & 2 & 3 & \dots & N\\
        P(1) & P(2) & P(3)&\dots&P(N)
    \end{matrix}\right)
\end{align}
举例
\begin{align}
    \hat{P}=\left(\begin{matrix}
        1&2&3&4&5\\
        4&1&2&5&3
    \end{matrix}\right)
\end{align}
可以表示为$P(1)=4,P(2)=1$。
对换(特殊的置换):
\begin{align}
    \hat{P}=\left(\begin{matrix}
        1&2&\dots&j&\dots&k&\dots&N\\
        1&2&\dots&k&\dots&j&\dots&N
    \end{matrix}\right)
\end{align}
我们将这种特殊的置换记作(j,k),当k=j+1时,对换(j,j + 1)也被称为相邻对换或轮换。
















\end{document}
