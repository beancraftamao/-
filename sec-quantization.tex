% -*- coding: utf-8 -*-
% !TEX program = xelatex
\documentclass[12pt, a4paper, oneside, UTF8]{ctexbook}
\usepackage{amsmath,xeCJK,hyperref,xcolor,comment,indentfirst,titlesec}
\usepackage{geometry}
\setCJKmainfont{KaiTi}
\setmainfont{Times New Roman}
\titleformat{\chapter}{\centering\bfseries\zihao{2}}{第\,\thechapter\,章}{1em}{}
\titleformat{\section}{\bfseries\zihao{3}}{\thesection}{1em}{}
\titleformat{\subsection}{\bfseries\zihao{4}}{\thesubsection}{1em}{}

\geometry{
    a4paper,
    left=3cm,
    right=3cm,
    top=3cm,
    bottom=1cm,
    headsep=1cm,
    footskip=1cm,
}
\begin{document}

\title{重生之每天坚持学不会从入学到入土\\组内自用手搓讲义}
\author{阿毛}
\date{\today}
\maketitle

\chapter{二次量子化}
二次量子化的形式可以从高量角度理解,也可以直接从场论出发,由于未学习场论,本章采用高量的语言。

\section{速通本章}
波函数语言在描述量子体系的时候是自洽的,但是在多体系统中,通常会出现
$\sum_N \int\Psi_N^{\dag}(r)\Psi_N(r) d(\vec{r})$的情况,共有$N!\times N!$项,计算量巨大。相较之下,
粒子数表示的表达非常简洁,且也能在多体系统中描述体系的变化(粒子的产生与湮灭)。
无论是费米子系统还是玻色子系统,都可以从波函数表示转化为粒子数表示
,这就是所谓的二次(历史上时间的第二次)量子化。

单体力学量算符$\hat{F}$(例如动量)在粒子数表象下可表示为:
\begin{eqnarray*}
    \hat{F}=\sum_{\alpha,\beta}f_{\alpha\beta}\hat{a}^{\dag}_\alpha\hat{a}_\beta,
\end{eqnarray*}
其中
\begin{eqnarray*}
    f_{\alpha\beta}=\langle\psi_\alpha|\hat{f}|\psi_\beta\rangle=\int dq \psi^{*}_\alpha(q)\hat{f}(q)\psi_\beta(q),
\end{eqnarray*}
而两体力学量算符
\begin{align}
    \hat{G}=&\sum_{a<b}\hat{g}(a,b)
\end{align}
例如:
\begin{align}
    V(q_1,q_2,q_N)=&\sum_{i<j}\frac{e^2}{|\mathbf{r}_i-\mathbf{r}_j|}=
    \frac{1}{2}\sum_{i,j}\frac{e^2}{|\mathbf{r}_i-\mathbf{r}_j|}
\end{align}
在粒子数表象中可以写为如下形式:
\begin{align}
    \hat{G}=\frac{1}{2}\sum_{\alpha'\beta'}\sum_{\alpha\beta}
    g_{\alpha'\beta',\alpha\beta}\hat{a}^{\dag}_{\alpha'}\hat{\alpha}^{\dag}_{\beta'}\hat{a}_\beta\hat{a}_\alpha
\end{align}
其中
\begin{eqnarray}
    g_{\alpha'\beta',\alpha\beta}=\int dq_1 dq_2 \varphi^*_{\alpha'}(q_1)\varphi^*_(q_2)\hat{g}(q_1,q_2)\varphi_\alpha(q_1)\varphi_\beta(q_2).
\end{eqnarray}
波函数$\Psi$则可以表示为粒子数的态$|n\rangle$,其中n是某个态的粒子数。
关于产生湮灭算符的作用,后面会提到,需要注意力学量作用在态上时,需要遵循火车到站的规则(老田的课里有),即算符作用后,存在一个置换的关系。


\section{数学准备}
首先明确本讲义目标:快速上手理解课题组工作中涉及的概念,如需要完整学习,详情参考B站田光善老师高等量子力学课程。
\href{https://www.bilibili.com/video/BV1qk4y1173i?p=1&vd_source=cd4587e54f9d57093a8dbe5712bb9c5e}
{\textcolor{blue}{网页链接}}

\subsection{置换群}
群论在物理学中的应用很广泛,但此处并不深入讨论,只需要把群操作视为一种算符,以方便理解。如群的定义,群的性质等,此处不说明。

置换定义如下:
\begin{align}
    \hat{P}=\left(\begin{matrix}
        1 & 2 & 3 & \dots & N\\
        P(1) & P(2) & P(3)&\dots&P(N)
    \end{matrix}\right)
\end{align}
举例
\begin{align}
    \hat{P}=\left(\begin{matrix}
        1&2&3&4&5\\
        4&1&2&5&3
    \end{matrix}\right)
\end{align}
可以表示为$P(1)=4,P(2)=1$。

对换(特殊的置换):
\begin{align}
    \hat{P}=\left(\begin{matrix}
        1&2&\dots&j&\dots&k&\dots&N\\
        1&2&\dots&k&\dots&j&\dots&N
    \end{matrix}\right)
\end{align}
我们将这种特殊的置换记作(j,k),当k=j+1时,对换(j,j + 1)也被称为相邻对换或轮换。

全同粒子波函数,可以写为$\varPsi_{\left(k_1,k_2,k_3,\dots,k_N\right)}(q_1,q_2,q_3,\dots,q_N)$
其中$k_N$为允许粒子存在的态,$q_N$为粒子的自由度。

\subsection{全同粒子波函数-以费米系统为例}
在全同粒子体系中,以两粒子体系为例,如果假设相互作用势为0,其哈密顿量可以写为如下形式
\begin{eqnarray}
    \hat{H}=\frac{\hat{p}_1^2}{2m}\otimes \hat{I}_2+\hat{I}_1\otimes\frac{\hat{P}_2^2}{2m}
\end{eqnarray}
对于一个单粒子,$\psi_k(\vec{r},s)=\frac{1}{\sqrt{V}}e^{i\vec{k} \cdot \vec{r}}\varphi(s)$
如果把自由度s理解为自旋(分别取两个态自旋向上,向下),双粒子体系可以写为如下形式
\begin{eqnarray}
    \psi_{k_1,k_2}(\vec{r}_1,\vec{r}_2)=\frac{1}{\sqrt{V}}e^{i\vec{k_1} \cdot \vec{r}_1}
    \left(\begin{matrix}
        1\\0
    \end{matrix}\right)
    \otimes\frac{1}{\sqrt{V}}e^{i\vec{k}_2\cdot\vec{r}_2}
    \left(\begin{matrix}
        0\\1
    \end{matrix}\right)
\end{eqnarray}
但此时波函数并没有满足费米统计规律,于是需要构造(凑)费米波函数:
\begin{align}
    \varPsi^A_{\vec{k}_1,\vec{k}_2}(\vec{r_1},\vec{r_2})=\psi_{k_1,k_2}(\vec{r}_1,\vec{r}_2)-\psi_{k_2,k_1}(\vec{r}_1,\vec{r}_2)
\end{align}
此时才满足
\begin{eqnarray}
    \varPsi^A_{\vec{k}_1,\vec{k}_2}(\vec{r_1},\vec{r_2})=-\varPsi^A_{\vec{k}_1,\vec{k}_2}(\vec{r_2},\vec{r_1})
\end{eqnarray}
即交换粒子位置,波函数多出一个负号。

初量的知识可知,全同粒子体系波函数$\psi(q_1,q_2,\cdots,q_N)=\sum_l C_l \psi_l(q_1,q_2,\cdots,q_N)$,
$\psi_l$满足

(1)正交归一

(2)体现体系由N个单粒子组成

(3)满足全同粒子体系的统计规律

对于费米子,可以定义
\begin{eqnarray}
    &&\psi_l(q_1,\cdots,q_i,\cdots,q_j,\cdots,q_N)\\\nonumber
    &=&\psi_{k_1,k_2,\cdots\cdots,k_N}(q_1,\cdots,q_i,\cdots,q_j,\cdots,q_N)\\\nonumber
    &=&D\sum_P (-1)^P \hat{P}\left(\varphi_{k_1}(q_1)\varphi_{k_2}q_2\cdots\cdots\varphi_{k_N}(q_N)\right)\\\nonumber
    &=&D\sum_P (-1)^P \varphi_{k_1}\left(q_{P(1)}\right)\cdots\cdots\varphi_{k_N}\left(q_{P(N)}\right) 
\end{eqnarray}
其中D为归一化系数,可以由排列组合得到,假设存在N个态,每个态存在$n_{k_N}$个粒子。
(注:能自己推可以尝试理解)

在费米体系中
\begin{eqnarray}
    \tilde{D}=\sqrt{\frac{n_{k_1}!n_{k_2}!\cdots\cdots n_{k_N}!}{N!}}=\frac{1}{\sqrt{N!}} 
\end{eqnarray}
且在上述波函数中,$k_1\neq k_2\neq \cdots \neq k_N$。(完整证明见高量讲义1-P10)

对该费米体系波函数作用一个对换
\begin{eqnarray}
    &&(i,j)\psi_{k_1,\cdots,k_N}(q_1,\cdots\cdots,q_N)\\\nonumber
    &=&\frac{1}{\sqrt{N!}}\sum_{\hat{P}}(-1)^{\hat{P}}(i,j)\hat{P}(\varphi_{k_1}(q_1)\cdots\cdots\varphi_{k_N}(q_N))\\\nonumber
    &=&\frac{1}{\sqrt{N!}}\sum_{\hat{P}}(-1)^{\hat{P}+1}(-1)[(i,j)\hat{P}](\varphi_{k_1}(q_1)\cdots\cdots\varphi_{k_N}(q_N))
\end{eqnarray}
令$(i,j)\hat{P}=\hat{P}'$,对换后$\hat{P}$和$\hat{P}'$必定具有不同的奇偶性。因此$(-1)^{\hat{P}'}=(-1)^{\hat{P+1}}$,上式可转化为
\begin{eqnarray}
    &&\frac{1}{\sqrt{N!}}\sum_{\hat{P}'}(-1)^{\hat{P}'}(-1)\hat{P'}(\varphi_{k_1}(q_1)\cdots\cdots\varphi_{k_N}(q_N))\\\nonumber
    &=&-\psi_{k_1,\cdots,k_N}(q_1,\cdots\cdots,q_N)
\end{eqnarray}
即多粒子体系,对换后体系的奇偶性改变,与两粒子一样成立。
对于这样的波函数,我们称其满足Dirac-Fermi统计规律。

而满足Dirac-Fermi统计规律的多体波函数可以写为如下形式:
\begin{align}
    \psi_{k_1,\cdots,k_N}(q_1,\cdots\cdots,q_N)=\frac{1}{\sqrt{N!}}\left|\begin{matrix}
        \phi_{k_1}(q_1)&\phi_{k_2}(q_1)&\cdots &\phi_{k_N}(q_1)\\
        \phi_{k_1}(q_2)&\phi_{k_2}(q_2)&\cdots &\phi_{k_N}(q_2)\\
        \vdots &\vdots&\vdots&\vdots\\
        \phi_{k_1}(q_N)&\phi_{k_2}(q_N)&\cdots &\phi_{k_N}(q_N)
    \end{matrix}\right|
\end{align}
该表达式被称为Slater行列式。

\section{粒子数表象}
\subsection{态指标的转化}
我们注意到,在任意基向量$\psi_{k_1,\cdots,k_N}(q_1,\cdots\cdots,q_N)$中,最重要的是单粒子态$k_1,\cdots,k_N$
出现的次数,于是我们引入记号
\begin{align}
    |n_{k_1},n_{k_2},\cdots,n_{k_N}\rangle
\end{align}
来表示上述的态。
注意对于玻色子情况,$|n_{k_1},n_{k_2}\rangle=|n_{k_2},n_{k_1}\rangle$,而费米子则不是。
即,波函数语言到粒子数语言,是一一对应的,而粒子数语言倒回去却不是。

为了解决该问题,我们引入真空态$|vac\rangle\equiv|0\rangle$
且要求真空态$\langle 0|0\rangle=1$,$\langle 0|n_k=1\rangle=0$,即与坐标表象一致,满足正交归一。

(1)定义产生算符$\hat{a}_k^{\dag}|0\rangle=|n_k=1\rangle$,同时要求$\langle n_k=1|n_k=1\rangle=1$。

(2)定义湮灭算符$\hat{a}_k|0\rangle=0 , \hat{a}_k|n_k=1\rangle=|0\rangle$。

如果满足恒等式
\begin{eqnarray}
    \langle\psi_1|\hat{A}|\psi_2\rangle=\langle\hat{A}^{\dag}\psi_1|\psi_2\rangle
\end{eqnarray}
则我们认为算符$\hat{A},\hat{A}^{\dag}$是共轭的,可以证明,上述产生湮灭算符也是共轭的。

对于费米子体系,我们用$\hat{C}^{\dag}_k$和$\hat{C}_k$来表示费米子的产生湮灭。

(1)同一个单粒子态上,不能有超过一个费米子
,即:
\begin{eqnarray}
    \hat{C}^{\dag}_k\hat{C}^{\dag}_k|0\rangle=\hat{C}^{\dag}_k|n_k=1\rangle=0
\end{eqnarray}
因此我们要求
\begin{eqnarray}
    \hat{C}^{\dag}_k\hat{C}^{\dag}_k=0
\end{eqnarray}
同时因为算符是共轭的,因此$(\hat{C}^{\dag}_k\hat{C}^{\dag}_k)^\dag =\hat{C}_k\hat{C}_k=0$

(2)导出对易关系

定义
\begin{eqnarray}
    \left\{\hat{C}_k,\hat{C}_k^\dag \right\}=\hat{C}_k\hat{C}_k^\dag+\hat{C}_k^\dag\hat{C}_k
\end{eqnarray}
将其作用在$|0\rangle$上
\begin{eqnarray}
    \hat{C}_k^\dag\hat{C}_k|0\rangle+\hat{C}_k\hat{C}_k^\dag |0\rangle=\hat{I}|0\rangle
\end{eqnarray}
将其作用在$|n_k=1\rangle$上
\begin{eqnarray}
    \hat{C}_k^\dag\hat{C}_k|n_k=1\rangle+\hat{C}_k\hat{C}_k^\dag |n_k=1\rangle=\hat{I}|n_k=1\rangle
\end{eqnarray}
即该对易算子作用在任意态上,其本征值都是单位矩阵,故此其对易关系等于单位矩阵
\begin{eqnarray}
    \left\{\hat{C}_k,\hat{C}_k^\dag \right\}=\hat{I}
\end{eqnarray}

(3)不同算符的对易关系

考虑算符$\hat{C}_k^\dag\hat{C}_{k'}^\dag+\hat{C}_{k'}^\dag\hat{C}_k^\dag$,定义
\begin{eqnarray}
    \hat{C}_k^\dag\hat{C}_{k'}^\dag|0\rangle=|n_k=1,n_{k'}=1\rangle
\end{eqnarray}
因此有
\begin{eqnarray}
    \hat{C}_k^\dag\hat{C}_{k'}^\dag|0\rangle+\hat{C}_{k'}^\dag\hat{C}_{k}^\dag|0\rangle&=&|n_k=1,n_{k'}=1\rangle+|n_{k'}=1,n_k=1\rangle\\\nonumber
    &=&|n_k=1,n_{k'}=1\rangle-|n_k=1,n_{k'}=1\rangle\\\nonumber
    &=&0
\end{eqnarray}
即
\begin{eqnarray}
    \hat{C}_k^\dag\hat{C}_{k'}^\dag+\hat{C}_{k'}^\dag\hat{C}_k^\dag=\left\{\hat{C}_k^\dag,\hat{C}_{k'}^\dag\right\}=0
\end{eqnarray}
同样我们可以得到$\left\{\hat{C}_k^\dag,\hat{C}_{k'}\right\}=0$
\subsection{单体算符}
对于一个多体量子体系,力学量可以写成如下形式
\begin{eqnarray}
    \hat{F}=\sum_{i=1}^{N}\hat{f}(q_i)
\end{eqnarray}
对于可以如上展开的算符,当我们对其作内积,存在等式
\begin{eqnarray}
    (\psi_{n_{k_1}\cdots n_{k_N}},\hat{F}\psi_{n'_{k_1\cdots n_{k_N}}})=N(\psi_{n_{k_1}\cdots n_{k_N}},\hat{f}(q_1)\psi_{n'_{k_1\cdots n_{k_N}}})
\end{eqnarray}
其证明过程见田光善讲义1-p18

由于$\hat{F}$是单体算符,初态和末态最多相差一个单粒子态,否则矩阵元为零,上述矩阵元可以写为如下形式
\begin{eqnarray}
    &&N \frac{1}{N!}\sum_{\hat{P}}\sum_{\hat{P}'}(-1)^{\hat{P}} (-1)^{\hat{P}'}\int dq_1\cdots dq_N \\\nonumber
    &\times& \hat{P}(\cdots\varphi^*_{k_j}(q_j)\cdots)\hat{f}(q_1)\hat{P'}(\cdots\varphi_{k_k}(q_k)\cdots)
\end{eqnarray}
其中非零项有
\begin{eqnarray}
    &&\int dq_1\cdots dq_N \varphi^*_{k_1}(q_\alpha)\cdots\varphi^*_{k_j}(q_1)\cdots\boxed{\varphi^*_{k_k}}\cdots \varphi^*_{k_N}(q_\lambda)\hat{f}(q_1)\\\nonumber
    &\times&\varphi_{k_1}(q_\alpha)\cdots\boxed{\varphi_{k_j}}\cdots\varphi_{k_k}(q_1)\cdots\varphi_{k_N}(q_\lambda)\\\nonumber
    &=&\int dq_1\varphi^*_{k_j}(q_1)\hat{f}(q_1)\varphi_{k_k}(q_1)\equiv f_{jk}
\end{eqnarray}
此时,还需要确定置换P与P',我们有:
\begin{eqnarray}
    (-1)^{\hat{P}'}=(-1)^{\hat{P}}(-1)^{\sum_{i=j+1}^{k-1}}
\end{eqnarray}
证明过程见讲义1-p29

在固定j,k之后(j的初态没有粒子,k的初态有粒子,依然剩余N-1个粒子)体系剩余自由度可以进行配对的全排列,即$(N-1)!$
因此最后有
\begin{eqnarray}
    &&\left(\psi_{\cdots ,1_j,\cdots,0_k,\cdots},\hat{F}\psi_{\cdots,0_j,\cdots,1_k,\cdots}\right)\\\nonumber
    &=&N \frac{1}{N!}(-1)^{\sum_{i=j+1}^{k-1}n_i}f_{jk}=(-1)^{\sum_{i=j+1}^{k-1}}f_{jk}
\end{eqnarray}
在粒子数表象中
\begin{eqnarray}
    &&\langle\cdots,n_k=0,\cdots,n_j=1,\cdots|\hat{F}|\cdots,n_j=1,\cdots,n_k=1,\cdots\rangle\\\nonumber
    &=&\sum_{\alpha\beta}f_{\alpha\beta}\langle\cdots,n_k=0,\cdots,n_j=1,\cdots|\hat{C}^\dag_\alpha\underbrace{\hat{C}_\beta |\cdots,n_j=1,\cdots,n_k=1},\cdots\rangle\\\nonumber
    &=&\sum_{\alpha\beta}f_{\alpha\beta}(-1)^{\sum_{i=1}^{k-1}n_i}\delta_{\beta k}\langle\cdots,n_k=0,\cdots,n_j=1,\cdots|\underbrace{\hat{C}^\dag_\alpha|\cdots,n_j=1},\cdots,n_k=0,\cdots\rangle\\\nonumber
    &=&\sum_{\alpha\beta}f_{\alpha\beta}(-1)^{\sum_{i=1}^{k-1}n_i}\delta_{\beta k}(-1)^{\sum_{i=1}^{j-1}n_i}\delta_{\alpha j}\\\nonumber
    &=&f_{jk}(-1)^{\sum_{i=j+1}^{k-1}n_i}
\end{eqnarray}
两种语言给出的结果一致。

\subsection{两体算符}
当讨论粒子的相互作用势时,我们会遇到两体算符
\begin{eqnarray}
    \hat{G}=\sum_{a<b}\hat{g}(a,b)
\end{eqnarray}
我们首先考虑对角元的情况
\begin{eqnarray}
    &&(\psi_{\cdots 0_i\cdots 0_j \cdots 1_k \cdots 1_l,\cdots},\hat{G}\psi_{\cdots 1_i\cdots 1_j \cdots 0_k \cdots 0_l,\cdots})\\\nonumber
    &=&C_N^2(\psi_{\cdots 0_i\cdots 0_j \cdots 1_k \cdots 1_l,\cdots},\hat{g}(q_1,q_2)\psi_{\cdots 1_i\cdots 1_j \cdots 0_k \cdots 0_l,\cdots})\\\nonumber
    &=&\frac{N(N-1)}{2}\frac{1}{N!}\sum_P\sum_{P'}(-1)^{\hat{P}}(-1)^{\hat{P}'}\int dq_1 \cdots dq_N\\\nonumber
    &\times&\hat{P}\left(\varphi^*_{k_1}(q_1)\cdots\varphi^*_{k_N}(q_N)\right)
    g(q_1,q_2)\hat{P'}\left(\varphi_{k_1}(q_1)\cdots\varphi_{k_N}(q_N)\right)
\end{eqnarray}
非零的项为
\begin{eqnarray}
    &&\int dq_1\cdots dq_N \varphi^*_{k_1}(q_\alpha)\cdots\varphi^*_{k_k}(q_1)\cdots\varphi^*_{k_s}(q_2)\cdots\varphi^*_{k_N}(q_\lambda)\\\nonumber
    &\times& g(q_1,q_2)\varphi_{k_1}(q_\alpha)\cdots\varphi_{k_k}(q_1)\cdots\varphi_{k_s}(q_2)\cdots\varphi_{k_N}(q_\lambda)
\end{eqnarray}
或者
\begin{eqnarray}
    &&\int dq_1\cdots dq_N \varphi^*_{k_1}(q_\alpha)\cdots\varphi^*_{k_k}(q_2)\cdots\varphi^*_{k_s}(q_1)\cdots\varphi^*_{k_N}(q_\lambda)\\\nonumber
    &\times& g(q_1,q_2)\varphi_{k_1}(q_\alpha)\cdots\varphi_{k_k}(q_1)\cdots\varphi_{k_s}(q_2)\cdots\varphi_{k_N}(q_\lambda)
\end{eqnarray}
前者记为
\begin{eqnarray}
    g_{ks,ks}\equiv\int dq_1dq_2\varphi^*_{k_k}(q_1)\varphi^*_{k_s}(q_2)g(q_1,q_2)\varphi_{k_k}(q_1)\varphi_{k_s}(q_2)
\end{eqnarray}
后者以同样形式可以记为$g_{ks,sk}=g_{sk,ks}$,前者被称为直接积分,后者被称为交换积分。
对于前者,因为轮换的次数完全相同,即$\hat{P}=\hat{P}'$,后者轮换次数多一个$-1$,由此可得
\begin{eqnarray}
    \hat{G}&=&\frac{N(N-1)}{2}\frac{1}{N!}\sum_{g\neq 0}2[g_{ks,ks}-g_{sk,ks}]n_{k_k}\cdot n_{s_k}\\\nonumber
    &=&\frac{N(N-1)}{2}\frac{1}{N!}(N-2)! 2[g_{ks,ks}-g_{sk,ks}]n_{k_k}\cdot n_{s_k}
\end{eqnarray}
上式中若态$\varphi_{k_k}$和$\varphi_{k_s}$不出现在态$\varphi_{n_{k_1},n_{k_2}\cdots}$时,该项贡献为0,此时
\begin{eqnarray}
    \hat{G}=\sum_{k,s}n_{k_k}n_{k_s}(g_{ks,ks}-g_{ks,sk})
\end{eqnarray}

在粒子数表象中进行验证
\begin{eqnarray}
    &&\langle\cdots,n_{k_2},n_{k_1}|\hat{G}|n_{k_1},n_{k_2},\cdots\rangle\\\nonumber
    &=&\frac{1}{2}\sum_{\alpha\beta}\sum_{\alpha'\beta'}g_{\alpha'\beta',\alpha\beta}\langle\cdots,n_{k_2},n_{k_1}|\hat{C}^{\dag}_{\alpha'}\hat{C}^{\dag}_{\beta'}\hat{C}_{\beta}\hat{C}_{\alpha}|n_{k_1},n_{k_2},\cdots\rangle\\\nonumber
    &=&\frac{1}{2}\sum_{\alpha'\neq\beta'}\sum_{\alpha\neq\beta}\langle\cdots,n_{k_2},n_{k_1}|\hat{C}^{\dag}_{\alpha'}\hat{C}^{\dag}_{\beta'}\hat{C}_{\beta}\hat{C}_{\alpha}|n_{k_1},n_{k_2},\cdots\rangle\delta_{\beta\beta'}\delta_{\alpha\alpha'}\\\nonumber
    &&+\frac{1}{2}\sum_{\alpha'\neq\beta'}\sum_{\alpha\neq\beta}\langle\cdots,n_{k_2},n_{k_1}|\hat{C}^{\dag}_{\alpha'}\hat{C}^{\dag}_{\beta'}\hat{C}_{\beta}\hat{C}_{\alpha}|n_{k_1},n_{k_2},\cdots\rangle\delta_{\alpha\beta'}\delta_{\beta\alpha'}\\\nonumber
    &=&\frac{1}{2}\sum_{\alpha\neq\beta}\langle\cdots,n_{k_2},n_{k_1}|\hat{C}^{\dag}_{\alpha}\hat{C}^{\dag}_{\beta}\hat{C}_{\beta}\hat{C}_{\alpha}|n_{k_1},n_{k_2},\cdots\rangle\\\nonumber
    &&+\frac{1}{2}\sum_{\alpha\neq\beta}\langle\cdots,n_{k_2},n_{k_1}|\hat{C}^{\dag}_{\beta}\hat{C}^{\dag}_{\alpha}\hat{C}_{\beta}\hat{C}_{\alpha}|n_{k_1},n_{k_2},\cdots\rangle\\\nonumber
    &=&\frac{1}{2}\sum_{\alpha\neq\beta}\langle\cdots,n_{k_2},n_{k_1}|\hat{C}^{\dag}_{\alpha}\hat{C}_{\alpha}\hat{C}^{\dag}_{\beta}\hat{C}_{\beta}|n_{k_1},n_{k_2},\cdots\rangle\\\nonumber
    &&\boxed{-}\frac{1}{2}\sum_{\alpha\neq\beta}\langle\cdots,n_{k_2},n_{k_1}|\hat{C}^{\dag}_{\beta}\hat{C}_{\beta}\hat{C}^{\dag}_{\alpha}\hat{C}_{\alpha}|n_{k_1},n_{k_2},\cdots\rangle\\\nonumber
\end{eqnarray}
到这一步,态指标内的算符可以转为粒子数算符$\hat{N}_\alpha\hat{N}_\beta$,于是我们有
\begin{eqnarray}
    &&\frac{1}{2}\sum_{\alpha\neq\beta}(g_{\alpha\beta,\alpha\beta}-g_{\alpha\beta,\beta\alpha})n_\alpha n_\beta\\\nonumber
    &=&\sum_{(\alpha,\beta)}(g_{\alpha\beta,\alpha\beta}-g_{\alpha\beta,\beta\alpha})n_\alpha n_\beta
\end{eqnarray}
与波函数语言结果一致

在两体算符中还存在如下非对角元
\begin{eqnarray}
    &&(\psi_{\cdots 0_i\cdots 0_j \cdots 1_k \cdots 1_l,\cdots},\hat{G}\psi_{\cdots 1_i\cdots 1_j \cdots 0_k \cdots 0_l,\cdots})\\\nonumber
    &=&C_N^2(\psi_{\cdots 0_i\cdots 0_j \cdots 1_k \cdots 1_l,\cdots},\hat{g}(q_1,q_2)\psi_{\cdots 1_i\cdots 1_j \cdots 0_k \cdots 0_l,\cdots})\\\nonumber
    &=&\frac{N(N-1)}{2}\frac{1}{N!}\sum_P\sum_{P'}(-1)^{\hat{P}}(-1)^{\hat{P}'}\int dq_1 \cdots dq_N\\\nonumber
    &\times&\hat{P}\left(\varphi^*_{k_1}(q_1)\cdots \boxed{\varphi^*_{k_i}} \cdots \boxed{\varphi^*_{k_j}}\cdots\varphi^*_{k_k}(q_k)\cdots \varphi^*_{k_l}(q_l)\cdots\varphi^*_{k_N}(q_N)\right)\\\nonumber
    &\times&g(q_1,q_2)\hat{P'}\left(\varphi_{k_1}(q_1)\cdots \varphi_{k_i}(q_i) \cdots \varphi_{k_j}(q_j)\cdots \boxed{\varphi_{k_k}}\cdots \boxed{\varphi_{k_l}}\cdots \varphi_{k_N}(q_N)\right)
\end{eqnarray}
非对角元的计算量较大,这里不详细证明,可以参考田光善讲义1-P32,同样的,非对角元形式依然能与粒子数语言一一对应。

\chapter{紧束缚模型}
紧束缚近似是研究固体中电子结构常用的理论模型,该模型假设在材料中大部分的电子都紧紧的束缚在原子核周围,只有少部分的
电荷可以自由的在格点之间跳跃,紧束缚这个名词也是由此而来。在该近似图象下,晶格自身的势能被视为一个场,我们可以用
化学势$\mu$以及场算符来表示,而电子之间的跃迁则用跳跃项来描述,即通常情况下,我们可以将总的哈密顿量写为这两部分
的和,下面给出式子:
\begin{eqnarray}
	\mathbf{H}=-\mathbf{t_{ij}}\sum_{\langle\mathbf{ij}\rangle,\sigma}(\mathbf{C^{\dag}_{i\sigma}C_{j\sigma}+h.c.})
	+\mu\sum_{i,\sigma} C^{\dag}_{i\sigma}C_{i\sigma},
\end{eqnarray}
其中$C^{\dag}_{i\sigma}$表示在格点i的位置产生一个自旋为$\sigma$的电子,$C_{j\sigma}$表示在格点j的位置湮灭一个
自旋为$\sigma$的电子,这在计算中可以等价为j格点上的电子跳跃到了i格点,$\mathbf{t}_{ij}$表示电子发生跳跃的强度,
该强度会受到化学键或者CDW流的强度的影响,需要注意的是,一般在计算中,考虑最近邻的跳跃项就能充分描述该体系的物理性质,
如果次近邻的子晶格存在较大的干扰或者散射,那么需要额外加上次近邻的跳跃项来完善模型。
第二项中,$C^{\dag}_{i\sigma}C_{i\sigma}$又可以看作是粒子数
算符$\hat{N}_{\sigma}=C^{\dag}_{i\sigma}C_{i\sigma}$,即分别对自旋和粒子数求和,得到总的势能。
\section{简单晶格中的跳跃项}

\subsection{最近邻项}

\subsection{次近邻项}

\section{蜂窝晶格中的跳跃项}

\section{kagome晶格中的跳跃项}

\chapter{BCS超导哈密顿量}
\section{推导过程}
\subsection{玻戈留波夫变换与BdG方程}
	BdG方程中的B,指的就是玻戈留波夫(Bogoliubov),dG则指的是de Genns。该方程用于描述费米子体系下的超导系统,并通过正则
	变换,将非对角元的有效哈密顿量对角化。
	以超导BCS哈密顿量为例:
	\begin{eqnarray}
		\mathbf{H}=\sum_{k}\varepsilon_{k}(C^{\dag}_{k \uparrow} C_{k \uparrow}+C^{\dag}_{-k \downarrow} C_{-k \downarrow})-
		\sum_{k}\Delta(C^{\dag}_{k \uparrow}C^{\dag}_{-k \downarrow}+C_{-k \downarrow}C_{k \uparrow})+\frac{\Delta^2}{V_0},
		\label{eqn:2-4}	
	\end{eqnarray}
	做Bogoliubov变换:
	\begin{align}
		C_{k \uparrow} =& u_k \alpha_{k \uparrow}+v_k \alpha^{\dag}_{-k \downarrow} \nonumber\\
		C^{\dagger}_{k \uparrow} =& u_k \alpha^{\dag}_{k \uparrow}+ v_k \alpha_{-k \downarrow}\nonumber\\
		C_{-k \downarrow} =& u_k \alpha_{-k \downarrow}-v_k \alpha^{\dag}_{k \uparrow}\nonumber\\ 
		C^{\dagger}_{-k \downarrow} =& u_k \alpha^{\dag}_{-k \downarrow}+ v_k \alpha_{k \uparrow} ,
	\label{eqn:2-5}
	\end{align}
	将式\ref{eqn:2-5}代入式\ref{eqn:2-4},展开后化简可得
	\begin{align}
		\mathbf{H}=&\sum_{k}\left\{
			\left[\varepsilon _k\left(u_k^2-v_k^2\right)+2\Delta u_k v_k\right]\alpha^{\dag}_{k \uparrow}\alpha_{k \uparrow}
			+\left[\varepsilon_k\left(u_k^2-v_k^2\right)+2\Delta u_k v_K\right]\alpha^{\dag}_{-k \downarrow}\alpha_{k \uparrow}
			\right\}\nonumber\\
			\quad&+\sum_{k}\left\{
				\left[2\varepsilon_k u_k v_k - \Delta \left(u_k^2-v_k^2\right)\right]\alpha^{\dag}_{k \uparrow}\alpha_{-k \downarrow}
				+\left[2\varepsilon_k u_k v_k- \Delta \left(u_k^2-v_k^2\right)\right]\alpha_{-k \downarrow}
			\right\}\nonumber\\
			\quad&+2\sum_k \varepsilon_k v_k^2-2\Delta \sum_k u_k v_k+\frac{\Delta^2}{V_0},
	\end{align}
	这里要求非对角项的系数为零,即$2\varepsilon_k u_k v_k-\Delta(u_k^2-v_k^2)=0$。又因为费米系统中算符存在反对易关系,如下所示:
	\begin{align}
		\left\{a_{k \uparrow},a^{\dag}_{k^{'} \uparrow}\right\}=&u_k u_{k^{'}} \left\{C_{k\uparrow},C^{\dag}_{k^{'}\uparrow}\right\}
		+v_k v_{k^{'}}\left\{C_{-k\downarrow}^{\dag},C_{-k^{'}\downarrow}\right\} \nonumber\\
		=&\delta_{kk^{'}} \left(u_k^2+v_k^2\right) \nonumber\\
		=&\delta_{kk^{'}},
	\end{align}
	所以要求$u_k^2+v_k^2=1$,根据这两个条件可以解出:
	\begin{eqnarray}
		u_k^2=\frac{1}{2}\left(1+\frac{\varepsilon_k}{E_k}\right),\quad v_k^2=\frac{1}{2}\left(1-\frac{\varepsilon_k}{E_k}\right),
	\end{eqnarray}
	其中$E_k=\sqrt{\varepsilon_k^2+\Delta^2}$,最终可以解得哈密顿量:
	\begin{eqnarray}
		\mathbf{H}=\sum_{k}E_k \left(\alpha^{\dag}_{k\uparrow} \alpha_{k\uparrow}+\alpha^{\dag}_{-k\downarrow}\alpha_{-k\downarrow}\right)+E_{s}(0),
	\end{eqnarray}
	其中$E_s (0) =2\sum_k \varepsilon_k v_k^2-2\Delta\sum_k u_kv_k+\frac{\Delta^2}{V_0}$。

	\subsection{南部表示与格林函数}
	南部表示是二次量子化方法中常用的一种表示方法,该表示下,上述的BdG方程可以整理为很简洁的矩阵形式。依然以BCS超导哈密顿量
	为例,如下所示:
	\begin{align}
		\mathbf{H}=&\sum_{k}\varepsilon_{k}(C^{\dag}_{k \uparrow} C_{k \uparrow}+C^{\dag}_{-k \downarrow} C_{-k \downarrow})-
		\sum_{k}\Delta(C^{\dag}_{k \uparrow}C^{\dag}_{-k \downarrow}+C_{-k \downarrow}C_{k \uparrow})+\frac{\Delta^2}{V_0}\nonumber\\
		=&\sum_{k}\left[C^{\dag}_{k\uparrow},C_{-k\downarrow}\right]
		\left[
			\begin{matrix}
				\varepsilon_k & -\Delta\\
				-\Delta & -\varepsilon_k
			\end{matrix}
		\right]
		\left[
			\begin{matrix}
				C_{k\uparrow}\\
				C^{\dag}_{-k\downarrow}
			\end{matrix}
		\right]
		+\sum_k \varepsilon_k+\frac{\Delta^2}{V_0},
	\end{align}
	定义$\hat{\psi}^{\dag}_k=[C^{\dag}_{k\uparrow},C_{-k\downarrow}]$,
	则有$H=\sum_k \hat{\psi}^{\dag}_k \hat{H}_k \hat{\psi}_k+\sum_k \varepsilon_k+\frac{\Delta^2}{V_0}$。

	南部表示下,时域的松原格林函数为:
	\begin{eqnarray}
		\hat{G}\left(k,\tau \right)=-\langle T_\tau \hat{\psi}_k(\tau)\hat{\psi}^{\dag}_0 \left(0\right)\rangle,
	\end{eqnarray}
	可以由离散傅里叶变换转到频域的格林函数:
	\begin{eqnarray}
		\hat{G}(\mathbf{k},\tau)=\frac{1}{\beta}\sum_n e^{-i\omega_n \tau} \hat{G}(\mathbf{k},i\omega_n),
	\end{eqnarray}
	其中$\omega_n=(2n+1)\pi/\beta$,是费米子的松原频率。





\end{document}